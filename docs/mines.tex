\documentclass[12pt]{article}

\usepackage[ruled, vlined]{algorithm2e}
\usepackage{amsmath}
\usepackage{amssymb}
\usepackage{booktabs}
\usepackage{geometry}
\usepackage{parskip}

\geometry{a4paper}

\title{Set-based Minesweeper}
\author{Andrew Kritzler}




\begin{document}
\maketitle
\tableofcontents
\newpage


\section{Definition of \(G\)}

Minesweeper is usually played on a finite rectangular grid containing a predetermined number of mines.
Consider the following \(3 \times 4\) grid with 3 mines.
\[
    \begin{matrix}
        \texttt{0} & \texttt{1} & \texttt{\#} & \texttt{*} \\
        \texttt{1} & \texttt{2} & \texttt{3}  & \texttt{-} \\
        \texttt{X} & \texttt{@} & \texttt{-}  & \texttt{-} \\
    \end{matrix}
\]

\texttt{*} is a mine, \texttt{@} is a detonated mine, \texttt{\#} is a correct flag, \texttt{X} is an incorrect flag, and \texttt{-} is a plain tile.
The following table provides a description of every cell.
\begin{center}
    \begin{tabular}{*{6}{c}}
        \toprule
        \(x\) & \(y\) & Tile        & Mine?     & Revealed? & Flagged?  \\
        \midrule
        0     & 0     & \texttt{X}  &           &           & \(\top \) \\
        0     & 1     & \texttt{1}  &           & \(\top \) &           \\
        0     & 2     & \texttt{0}  &           & \(\top \) &           \\
        1     & 0     & \texttt{@}  & \(\top \) & \(\top \) &           \\
        1     & 1     & \texttt{2}  &           & \(\top \) &           \\
        1     & 2     & \texttt{1}  &           & \(\top \) &           \\
        2     & 0     & \texttt{-}  &           &           &           \\
        2     & 1     & \texttt{3}  &           & \(\top \) &           \\
        2     & 2     & \texttt{\#} & \(\top \) &           & \(\top \) \\
        3     & 0     & \texttt{-}  &           &           &           \\
        3     & 1     & \texttt{-}  &           &           &           \\
        3     & 2     & \texttt{*}  & \(\top \) &           &           \\
        \bottomrule                                                     \\
    \end{tabular}
\end{center}

Hopefully, the following transformation is self-explanatory.
For each entry in the matrix, \((x, y) \in M\) if and only if it is a mine.
Similarly, sets \(R\) and \(F\) are used for the remaining columns.
\[
    \begin{aligned}
        M & = \{(1, 0), (2, 2), (3, 2)\}                         \\
        R & = \{(0, 1), (0, 2), (1, 0), (1, 1), (1, 2), (2, 1)\} \\
        F & = \{(0, 0), (2, 2)\}
    \end{aligned}
\]

\(G = (M, R, F)\) is a minesweeper game, where \(M, R, F \subseteq \mathbb{Z}^2\) and \(R \cap F = \varnothing \).
Notice how \(G\) does not encode the dimensions of the grid; the grid is considered to be infinite.
An infinite game is impractical because computers have finite resources and the player could never win.
However, the simplicity of this model is useful for mathematical study.


\section{Dynamic Generation of \(M\)}

Let \(A : \mathbb{Z}^2 \to \mathcal{P}(\mathbb{Z}^2)\) be a function that outputs the set of adjacent coordinates.
\(A\) is a useful building block for more complex logic.
For example, \(|A(x, y) \cap M|\) counts the number of mines adjacent to \((x,y)\).
\[A(x, y) = (\{x - 1, x, x + 1\} \times \{y - 1, y, y + 1\}) \setminus \{(x, y)\} \]

Generating \(M\) statically is impossible.
An infinite board cannot be described in terms of its number of mines.
Instead, the board must be described in terms of its density of mines.
Let \(\gamma \) be a predetermined value representing the proportion of cells which are mines, where \(0 \leq \gamma \leq 1\).
For example, in a \(16 \times 30\) board with 99 mines, \(\gamma = 99 \div (16 \times 30) = 0.20625\).

A cell \((x, y)\) is \textit{inactive} if it and its adjacent cells have not been revealed, i.e., \((x, y) \notin R\) and \(A(x, y) \cap R = \varnothing \).
When the player reveals a cell, mines are randomly placed in the inactive cells within \(A(x, y) \cup \{(x, y)\} \).
Afterwards, all nine of these cells become active.
On the player's first turn when \(R = \varnothing \), no mines are generated.
Thus, the first tile revealed will always be \texttt{0}.

For notation purposes, let \(X\) be a standard uniform random variable, i.e., \(X \sim U(0, 1)\).
As a result, \(\Pr(X < \gamma ) = \gamma \).

\begin{algorithm}[H]
    \SetAlgoLined{}
    \DontPrintSemicolon{}
    \caption{Reveal}
    \eIf{\((x, y) \notin R \cup F\)}{
        \If{\(R \neq \varnothing \)}{
            \ForAll{\((u, v) \in A(x, y) \cup \{(x, y)\} \)}{
                \If{\((u, v) \notin R \wedge A(u, v) \cap R = \varnothing \)}{
                    \If{\(X < \gamma \)}{
                        \(M \gets M \cup \{(u, v)\} \)
                    }
                }
            }
        }
        \(R \gets R \cup \{(x, y)\} \) \\
        \Return{\(\top \)}
    }{
        \Return{\(\bot \)}
    }
\end{algorithm}


\end{document}
