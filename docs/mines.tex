\documentclass{article}

\usepackage{amsmath}
\usepackage{amssymb}
\usepackage{booktabs}

\title{Set-based Minesweeper}
\author{Andrew Kritzler}

\begin{document}
\maketitle
\tableofcontents
\newpage


\section{Definition of \(G\)}

Minesweeper is usually played on a finite rectangular grid containing a predetermined number of mines.
Consider the following \(3 \times 4\) grid with 3 mines.

\[
    \begin{matrix}
        \texttt{0} & \texttt{1} & \texttt{\#} & \texttt{*} \\
        \texttt{1} & \texttt{2} & \texttt{3}  & \texttt{-} \\
        \texttt{X} & \texttt{@} & \texttt{-}  & \texttt{-} \\
    \end{matrix}
\]

\texttt{*} is a mine, \texttt{@} is a detonated mine, \texttt{\#} is a correct flag, \texttt{X} is an incorrect flag, and \texttt{-} is a plain tile.
The following table provides a description of every cell.

\begin{center}
    \begin{tabular}{*{6}{c}}
        \toprule
        \(x\) & \(y\) & Tile        & Mine?     & Revealed? & Flagged?  \\
        \midrule
        0     & 0     & \texttt{X}  &           &           & \(\top \) \\
        0     & 1     & \texttt{1}  &           & \(\top \) &           \\
        0     & 2     & \texttt{0}  &           & \(\top \) &           \\
        1     & 0     & \texttt{@}  & \(\top \) & \(\top \) &           \\
        1     & 1     & \texttt{2}  &           & \(\top \) &           \\
        1     & 2     & \texttt{1}  &           & \(\top \) &           \\
        2     & 0     & \texttt{-}  &           &           &           \\
        2     & 1     & \texttt{3}  &           & \(\top \) &           \\
        2     & 2     & \texttt{\#} & \(\top \) &           & \(\top \) \\
        3     & 0     & \texttt{-}  &           &           &           \\
        3     & 1     & \texttt{-}  &           &           &           \\
        3     & 2     & \texttt{*}  & \(\top \) &           &           \\
        \bottomrule
    \end{tabular}
\end{center}

Hopefully, the following transformation is self-explanatory.
For each entry in the matrix, \((x, y) \in M\) if it is a mine.
Similarly, sets \(R\) and \(F\) are used for the remaining columns.

\[
    \begin{aligned}
        M & = \{(1, 0), (2, 2), (3, 2)\}                         \\
        R & = \{(0, 1), (0, 2), (1, 0), (1, 1), (1, 2), (2, 1)\} \\
        F & = \{(0, 0), (2, 2)\}
    \end{aligned}
\]

Formally, \(G = (M, R, F)\) is a minesweeper game, where \(M, R, F \subseteq \mathbb{Z}^2\) and \(R \cap F = \varnothing \).


\end{document}
